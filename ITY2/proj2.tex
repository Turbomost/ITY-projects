\documentclass[a4paper, 11pt, twocolumn]{article}

% Packages

\usepackage[czech]{babel}
\usepackage[left = 1.8cm, top = 2.5cm, text={18cm, 25cm}]{geometry}
\usepackage[T1]{fontenc}
\usepackage{hyperref}
\usepackage{amsmath, amsthm, amssymb, amsfonts}
\usepackage{xcolor,listings}
\usepackage[utf8]{inputenc}
\usepackage{stackrel}
\usepackage{times}

\begin{document}

\begin{titlepage}
\onecolumn
\begin{center}
\Huge
\textsc{Fakulta informačních technologií\\[0.4em]
Vysoké učení technické v~Brně}\\
\vspace{\stretch{0.382}}
 
\LARGE
Typografie a publikování\,--\,2. projekt\\[0.3em]
Sazba dokumentú a~matematických výrazů\\
\vspace{\stretch{0.618}}

\Large
\textbf{ 2021 \hfill Václav Valenta (xvalen29)}

\end{center} 
\end{titlepage}

\thispagestyle{plain}
\clearpage

\twocolumn
\section*{Úvod}
V této úloze si vyzkoušíme sazbu titulní strany, matematických vzorců, prostředí a dalších textových struktur obvyklých pro technicky zaměřěné texty (například rovnice (1) nebo Definice I na strant 1). Rovněž si vyzkoušíme používání odkazdů \verb=\ref= a \verb=\pageref=.

Na titulní straně je využito sázení nadpisu podle optického středu s využitím zlatého řezu. Tento postup byl probírán na přednášce. Dále je použito odřádkování se zadanou relativní velikostí 0.4\,em a 0.3\,em.

V případě, že budete potřebovat vyjádřit matematickou konstrukci nebo symbol a nebude se Vám dařit jej nalézt v samotném \LaTeX, doporučuji prostudovat možnosti balíku maker \AmS\--\LaTeX.

\section{Matematický text}
Nejprve se podíváme na sázení matematických symbolů a výrazů v plynulém textu včetně sazby definic a vět s vvyužitím balíku \verb=ansthn=. Rovněž použijeme poznámku pod čarou s použitím příkazu \verb=\footnote=. Někdy je vhodné použít konstrukci \verb=\mbox{ }=, která říká, že text nemá být zalomen.\\

\noindent
\textbf{Definice 1.} Rozšířšný zásobníkový automat \textit{(RZA) je definován jako sedmice tvaru
$A=(Q, \Sigma, \Gamma, \delta, q_{0}, Z_{0}, F)$,
kde:}

\begin{itemize}
\item $Q$ \textit{je konečná množin}a vnitřních (řídících) stavů,
\item $\Sigma$ \textit{je konečná} vstupní abeceda,
\item $\Gamma$ \textit{je konečná} zásobníková abeceda,
\item $\delta$ \textit{je} přechodová funkce $Q \times(\Sigma \cup\{\epsilon\}) \times \Gamma^{*} \rightarrow 2^{Q \times \Gamma^{*}}$
\item $q_{0} \in Q$ je počáteční stav, $Z_{0} \in \Gamma$  \textit{je} startovací symbol zásobníku $a F \subseteq Q$ \textit{je množina} koncových stavů.
\end{itemize}

Nechť $P=(Q, \Sigma, \Gamma, \delta, q_{0}, Z_{0}, F)$ je rozšírený zásobníkový automat.
\textit{Konfigurací} nazveme trojici $(q, w, \alpha) \in$ $Q \times \Sigma^{*} \times \Gamma^{*}$,
kde $q$ je aktuální stav vnitřního řízení, $w$ je dosud nezpracovaná část vstupního řetězce a
$\alpha=$ $Z_{i_{1}} Z_{i_{2}} \ldots Z_{i_{k}}$ je obsah zásobníku\footnote{$Z_{i_{1}}$ je vrchol zásobníku}.

\subsection {Podsekce obsahující větu a odkaz }
\textbf{Definice 2.}
Řetězec $w$ nad abecedou $\Sigma$ je přijat RZA 
\textit{A~jestliže} $(q_{0}, w, Z_{0})\stackrel[A]{*}{\vdash}(q_{F}, \epsilon, \gamma)$ \textit{pro nějaké} $\gamma \in \Gamma^{*}$ \textit{a}
$q_{F} \in F.$\textit{ Množinu }$L(A)=\{w \mid w$ \textit{je přijat RZA} $A\} \subseteq$ $\Sigma^{*}$ \textit{nazýváme} jazyk přijímaný RZA A.

Nyní si vyzkoušíme sazbu vět a důkazů opět s použitím balíku \verb=amsthm=.

\medskip
\noindent\textbf{Věta 1.} \textit{Třída jazyků, které jsou přijímány ZA, odpovídá} bezkontextovým jazykům.

\medskip
\noindent\textit{Důkaz.} V dùkaze vyjdeme z Definice 1 a 2 .

2\section{Rovnice a odkazy}
Složitější matematické formulace sázíme mimo plynulý text. Lze umístit několik výrazů na jeden řádek, ale pak je třeba tyto vhodně oddělit, například příkazem \verb=\quad=

\bigskip
\noindent$\sqrt[i]{x_{i}^{3}} \quad$ kde $x_{i}$ je $i$ -té sudé číslo splňující $\quad x_{i}^{x_{i}^{x^{2}}+2} \leq y_{i}^{x_{i}^{4}}$\\

V rovnici (\ref{eq1}) jsou využity tři typy závorek s různou explicitně definovanou velikostí.
\begin{equation}\label{eq1}
x=[\{[a+b] * c\}^{d} \oplus 2]^{3 / 2}\\
\end{equation}
\[y=\lim _{x \rightarrow \infty} \frac{\frac{1}{\log _{10} x}}{\sin ^{2} x+\cos ^{2} x}\]

V této větě vidíme, jak vypadá implicitná vysázení limity
$\lim _{n \rightarrow \infty} f(n)$
v normálním odstavci textu. Podobně je to i s dalšími symboly jako
$\prod_{i=1}^{n} 2^{i}$ či $\bigcap_{A \in \mathcal{B}} A$.
V případě vzorců 
$\lim\limits_{n \rightarrow \infty} f(n)$ a $\prod\limits_{i=1}^{n} 2^{i}$
jsme si vynutili méně úspornou sazbu příkazem \verb=\limits=

\begin{equation}\label{eq2}
\int\limits_{b}^{a} g(x) \mathrm{d} x=-\int\limits_{a}^{b} f(x) \mathrm{d} x
\end{equation}


\section{Matice}
Pro sázení matic se velmi často používá prostředí \verb=\array= a závorky ( \verb=\left=,  \verb=\right=)

\begin{align*}
\left(\begin{array}{ccc}
a-b & \widehat{\xi+\omega} & \pi \\
\overrightarrow{\mathbf{a}} & \stackrel{\leftrightarrow}{A C} & \hat{\beta}
\end{array}\right)=1 \Longleftrightarrow \mathcal{Q}=\mathbb{R}
\end{align*}
\begin{align*}
\mathbf{A}=\left\|\begin{array}{cccc}
a_{11} & a_{12} & \ldots & a_{1 n} \\
a_{21} & a_{22} & \ldots & a_{2 n} \\
\vdots & \vdots & \ddots & \vdots \\
a_{m 1} & a_{m 2} & \ldots & a_{m n}
\end{array}\right\|=\left|\begin{array}{cc}
t & u \\
v & w
\end{array}\right|=t w-u v
\end{align*}
Prostředí array \verb=\array= lze úspěšně využí i jinde.
\[\left(\begin{array}{l}n \\ k\end{array}\right)=\left\{\begin{array}{cl}0 & \text { pro } k<0 \text { nebo } k>n \\ \frac{n !}{k !(n-k) !} & \text { pro } 0 \leq k \leq n\end{array}\right.\]

\end{document}
