\documentclass[a4paper, 11pt]{article}

% Packages
\usepackage[czech]{babel}
\usepackage[left = 2cm, top = 3cm, text={17cm, 24cm}]{geometry}
\usepackage[utf8]{inputenc}
\usepackage[T1]{fontenc}
\usepackage{hyperref}
\usepackage{url}
\usepackage{csquotes}
\usepackage{expl3}
\bibliographystyle{czplain}

% Title page
\begin{document}
\begin{titlepage}
\begin{center}
\Huge\textsc{Vysoké učení technické v~Brně}\\
\huge\textsc{Fakulta informačních technologií}\\[0.4em]
\vspace{\stretch{0.382}}

\LARGE{Typografie a publikování\,--\,4. projekt}\\
\Huge{Citace}
\vspace{\stretch{0.618}}

\Large\today \hfill Václav Valenta (xvalen29)
\end{center} 
\end{titlepage}

% Page 1
\section*{Font}
\textit{Font je to, co budete obdivovat po dokončení práce, ale styl písma je nástroj, se kterými musíte zápasit, abyste svou práci zvládli.\cite{Felici:The_Complete_Manual_of_Typography}}
\bigskip

Pojem font je v typografii definován jako kompletní sada znaků abecedy jedné velikosti a jednotného stylu.\cite{wiki:Font}.\bigskip

Na počátku 13. století vznikl v Evropě nový sloh ve výtvarném umění, zvaný gotický. Tento sloh se uplatnil i v písmařství. Obměny a různé variace písma jsou mnohem početnější a bohatší. \cite{Miklova:Pismo_v_aranzerske_a_propagacnni_cinnosti}
Blackletter, také známý jako Old English, Gothic, nebo Fraktur, byl prvním fontem na světě. Tento styl získal uznání od mnoha lidí kvůli jeho dramatickým silným a tenkým tahům.\cite{Music_Mayhem:The_History_of_The_First_World_Font}

V první polovině 20. století písma ze starších let začínají předělávat do
počítačové podoby a v následující druhé polovině 20. století se jejich podoba obecně
zjednodušuje tak, aby splňovala svou funkčnost.\cite{Koprivova:Pismo_volny_soubor}
Primitivní počítačče z doby před prvním Macintoshem z roku 1984 nabízely jeden nezajímavý typ písma, který bylo se štěstím možné zménit do kurzivy. Najednou tu ale byl reálný výběr abeced, které se snažily replikovat zkušenost z reálného života.\cite{Garfield:Just_My_Type}

\bigskip
\bigskip


\section*{Kryptoměny}
\textit{Zlato bylo dlouhou dobu považováno za jednu z nejbezpečnějších dlouhodobých finančních investic. Během jednoho roku ale cena tohoto drahého kovu klesla přibližně o třetinu\cite{Muller:Poprask_kolem_Bitcoinu}}
\bigskip

Kryptoměna je typ digitální měny, lterá se opírá o asymetrickou kryptografii.\cite{wiki:Kryptomena}
Krytoměny fungují na velmi jednoduchém, ale důvtipném principu. Na jedné strané jsou finanční transakce a na druhé těžaři ověřujíccí tytro transakce a za úplatu dostávají ocenění v podobě digitálních mincí.\cite{Panek:Novy_usvit_tezby}

Na rozdíl od investic v tradičních měnách nejsou kryptoměny vydávány centrální bankou ani podporovány vládou; faktory jako jsou: měnová politika, inflace, či ekonomický růst, které obvykle ovlivňují hodnotu měny, se tedy na krytoměny nevztahují.\cite{Bloomenthal:What_Determines_the_Price_of_1_Bitcoin}

\bigskip
Navzdory předpokladům trh s kryptoměnami stále roste. Ačkoliv většina z nich je nevýznamných a jen minimum se skutečně stane využívanými, jejich počet je obrovský.\cite{Computerworld:Trh_kryptomen_stale_zaplavuji_nove_meny}

\newpage
\bibliography{literatura}
\end{document}